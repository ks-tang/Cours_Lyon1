\documentclass[varwidth = 17cm]{standalone}
\usepackage{ucblalgo}

\begin{document}

\begin{ucblalgo}
  \Entree{
    \begin{itemize}
      \item $v$ : une valeur à insérer
    \end{itemize}
  }
  \Algorithme {
    niveau courant est initialisé au niveau maximal de la skip liste \;
    créer un tableau précédentes aussi grand que le nombre de niveaux \;
    créer un curseur initialisé sur la cellule sentinelle \;
    \Tq{le niveau courant est positif ou nul}{
      \Tq{la cellule suivant le curseur au niveau courant existe et a une valeur plus petite}{
        avancer le curseur sur sa cellule suivante au niveau courant \;
      }
      sauvegarder le curseur dans le tableau de précédentes au niveau courant \;
      diminuer le niveau courant \;
    }
    mettre à 0 le niveau courant \;
    \Tq{le niveau courant est plus petit que le niveau maximal}{
      \uSi{la nouvelle cellule est sélectionnée pour le niveau courant}{
        insérer la nouvelle cellule entre la précédente du niveau courant et sa suivante au niveau courant
      }
      \Sinon{
        sortir de la fonction
      }
    }
    \Tq{la nouvelle cellule est sélectionnée pour un niveau supérieur}{
      ajouter un niveau à la cellule sentinelle et à la nouvelle cellule \;
      la suivante de la sentinelle sur ce niveau devient la nouvelle cellule \;
    }
  }
\end{ucblalgo}

\end{document}
