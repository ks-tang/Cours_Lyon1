\documentclass{standalone}
\usepackage{tikz}

\begin{document}

\tikzset{
  ufnode/.style={
    circle,
    draw,
    thick
  },
  ufedge/.style={
    -latex, 
    thick
  }
}

\begin{tikzpicture}
  \draw
  (0,1) node[ufnode] (n00) {0}
  (1,2) node[ufnode] (n10) {1}
  (2,1) node[ufnode] (n20) {2}
  (-1,0) node[ufnode] (n30) {3}
  (1,0) node[ufnode] (n40) {4}
  (1,-1) node[ufnode] (n50) {5}
  ;

  \draw[ufedge] (n00) -- (n10) ;
  \draw[ufedge] (n20) -- (n10) ;
  \draw[ufedge] (n30) -- (n00) ;
  \draw[ufedge] (n40) -- (n00) ;
  \draw[ufedge] (n50) -- (n40) ;
  \draw[ufedge] (n10) edge[loop above] (n10) ;

  \draw[ufedge] (4,1) -- (6.5,1) ;

  \draw
  (-0.5,1) +(9,0) node[ufnode] (n01) {0}
  (1,2)    +(9,0) node[ufnode] (n11) {1}
  (2.5,1)  +(9,0) node[ufnode] (n21) {2}
  (-1.5,0) +(9,0) node[ufnode] (n31) {3}
  (0.5,1)  +(9,0) node[ufnode] (n41) {4}
  (1.5,1)  +(9,0) node[ufnode] (n51) {5}
  ;

  \draw[ufedge] (n01) -- (n11) ;
  \draw[ufedge] (n21) -- (n11) ;
  \draw[ufedge] (n31) -- (n01) ;
  \draw[ufedge] (n41) -- (n11) ;
  \draw[ufedge] (n51) -- (n11) ;
  \draw[ufedge] (n11) edge[loop above] (n11) ;


\end{tikzpicture}

\end{document}
